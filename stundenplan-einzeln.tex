\cleardoublepage
\phantomsection
\pdfbookmark{
%(name)s
}{stundenplan\arabic{schuelernummer}}

\addtocounter{schuelernummer}{1}

\begin{center}
\textbf{\Large{\underline{Stundenplan von
%(name)s
}}}
\end{center}

\tagg{Donnerstag}

{\parskip=-1pt

%(nichtphysik0)s

%(nichtphysik1)s

%(kurs0)s

%(kurs1)s

%(nichtphysik2)s

%(nichtphysik3)s

\vspace{3ex}
}

\tagg{Freitag}

{\parskip=-1pt

%(nichtphysik4)s

%(nichtphysik5)s

%(kurs2)s

%(nichtphysik6)s

%(nichtphysik7)s

%(kurs3)s

%(kurs4)s

%(kurs5)s

%(nichtphysik8)s

%(nichtphysik9)s

\vspace{3ex}
}

%(betreuerpagebreak)s

\tagg{Samstag}

{\parskip=-1pt

%(nichtphysik10)s

%(nichtphysik11)s

%(kurs6)s

%(kurs7)s

%(nichtphysik12)s

%(nichtphysik13)s

%(nichtphysik14)s

%(nichtphysik15)s

%(nichtphysik16)s

\vspace{3ex}
}

%(betreuerkeinpagebreak)s

\tagg{Sonntag}

{\parskip=-1pt

%(nichtphysik17)s

%(nichtphysik18)s

%(kurs8)s

%(kurs9)s

%(nichtphysik19)s

%(nichtphysik20)s

%(nichtphysik21)s

\vspace{3ex}
}

\vspace{0.7cm}

Notfallnummern:\\
0152 264 838 60 Luis Ramos Henriques (Hauptorganisator)\\
0176 271 368 43 Manuel Bärenz (Stundenplan)

\begin{center}\Large Hast du Lust, die Anderen vom Seminar wiederzusehen?\end{center}

Dann komm doch einfach zum \textbf{Vereinstreffen in Leipzig, vom 3. bis zum 6. Januar}. (Du musst dafür kein Vereinsmitglied sein.) Neben \textbf{interessanten Vorträgen} einiger Vereinsmitglieder zu ihrer eigenen Forschung sind eine \textbf{Stadtführung} und jede Menge \textbf{Spiele und Spaß} geplant. Schau in einem Monat einfach noch mal auf die Website. Wir freuen uns, wenn du dabei bist!

\vspace{0cm}

\begin{center}
	\includegraphics[width=12cm]{../../Gebaeudeplan_Physik.png}
\end{center}

\cleardoublepage
